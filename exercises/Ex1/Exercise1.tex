\documentclass[a4paper,12pt]{scrartcl}

\usepackage{bm,amsmath,url,graphicx}
\usepackage{palatino}
\usepackage{color, xcolor}
\usepackage{listings}


\newcommand{\n}{{\bf n}}
\newcommand{\h}{{\bf h}}
\newcommand{\x}{{\bf x}}
\newcommand{\HH}{{\bf H}}
\newcommand{\thb}{{\boldsymbol{\theta}}}
\newcommand{\python}{{\fbox{\texttt{\bfseries python}}\quad}}

\renewcommand{\familydefault}{\rmdefault}


\begin{document}
\section*{\bf SGN-41007 Pattern Recognition and Machine Learning}
\emph{Exercise Set 1: January 11 - 12, 2017}
\bigskip
\sloppy

\lstdefinestyle{mystyle}{
  belowcaptionskip=1\baselineskip,
  breaklines=true,
  frame=single,
  xleftmargin=\parindent,
  language=Python,
  showstringspaces=false,
  basicstyle=\ttfamily,
  keywordstyle=\bfseries\color{green!40!black},
  commentstyle=\itshape\color{purple!40!black},
  identifierstyle=\color{blue},
  stringstyle=\color{orange},
  moredelim=**[is][\color{red}]{@}{@},
}

\lstset{language=Python,style=mystyle} 


\noindent
Exercises consist of both pen\&paper and computer assignments.
Pen\&paper questions are solved at home before exercises, while
computer assignments are solved during exercise hours. The
computer assignments are marked by text \python

\begin{enumerate}

\item \python \emph{Load CSV file into Python.}

Download the following file and extract the contents:
{\small
\begin{verbatim}
http://www.cs.tut.fi/courses/SGN-41007/exercises/locationData.zip
\end{verbatim}
}
\label{ex1}

\begin{enumerate}
	\item Read the file into numpy array using \verb|numpy.loadtxt| function.
Search for instructions using Google.
\item When loaded, check the array shape using \verb|numpy.shape|. It should be $600\times 3$.
\end{enumerate}

\item \python \emph{Plot the contents of the loaded matrix.}
\begin{enumerate}
\item Create a 2D plot of the first two columns of the matrix.
You need to import \verb+matplotlib.pyplot+ and state something like
\begin{verbatim}
plt.plot(<column 1>, <column 2>)
\end{verbatim}
\item Create a 3D plot of all 3 columns. You will need to create a special subplot 
for this purpose with 
\begin{verbatim}
ax = plt.subplot(1, 1, 1, projection = "3d")
\end{verbatim}
After this, the plotting is done as
\begin{verbatim}
plt.plot(<column 1>, <column 2>, <column 3>)
\end{verbatim}
\end{enumerate}

\item \python \emph{Basic image manipulation routines in Python.}
\begin{enumerate}
\item Download the following file to your local folder:
{\small
\begin{verbatim}
http://www.cs.tut.fi/courses/SGN-41007/exercises/oulu.jpg
\end{verbatim}
}
\item Load the image into numpy array using the \verb+imread+ function
of \verb+matplotlib.image+.
\footnote{Note: There are several other ways of doing this, such as:
\texttt{matplotlib.image.imread}, \texttt{scipy.ndimage.imread},\quad 
\texttt{PIL.Image.open} \quad or \quad \texttt{cv2.imread}}

\item Show the image on screen (\verb|matplotlib.pyplot.imshow|).
\item Print the image shape.
\item Compute the mean of the whole image.
\item Compute the means of the three color channels (R, G, B).
\item Apply the white tophat operator to the image (\verb|scipy.ndimage.morphology.white_tophat|)
and show the result on screen.

\end{enumerate}

\item \python \emph{Load Matlab data into Python.}
\label{ex4}
\begin{enumerate}
\item Download the following file to your local folder:
{\small
\begin{verbatim}
http://www.cs.tut.fi/courses/SGN-41007/exercises/twoClassData.mat
\end{verbatim}
}
\item Load the file contents into Python. This can be done as follows.
\begin{lstlisting}
>>> from scipy.io import loadmat
>>> mat = loadmat("twoClassData.mat")
\end{lstlisting}
This generates a \lstinline|dict| structure, whose elements can be accessed through their names.
\begin{lstlisting}
>>> print(mat.keys()) # Which variables mat contains?
['y', 'X', '__version__', '__header__', '__globals__']
>>> X = mat["X"] # Collect the two variables.
>>> y = mat["y"].ravel()
\end{lstlisting}
The function \verb+ravel()+ transforms y from $400\times 1$ matrix into a 400-length array.
In Python these are different things unlike Matlab.

\item The matrix X contains two-dimensional samples from two
classes, as defined by y. Plot the data as a scatter plot
like the picture below.
Hints:
\begin{itemize}
	\item You can access all class 0 samples from X as: \verb+X[y == 0, :]+.
	\item The samples can be plotted like: \verb+plt.plot(X[:, 0], X[:, 1], 'ro')+
\end{itemize}
\centerline{\includegraphics[width=0.6\textwidth]{twoClasses.pdf}}

\end{enumerate}
\eject 

\item \python \emph{Define a function.}
\begin{enumerate}
\item Implement a function that normalizes the data of question \ref{ex4} by
subtracting the mean and dividing by the standard deviation for each
column separately. The function call should be:
\begin{verbatim}
X_norm = normalize_data(X)
\end{verbatim}
\item Test your code with the following lines
\begin{verbatim}
X_norm = normalize_data(X)
np.mean(X_norm, axis = 0) # Should be 0
np.std(X_norm, axis = 0)  # Should be 1
\end{verbatim}


\end{enumerate}


\end{enumerate}

\end{document}          
